\documentclass[12pt,letterpaper]{article}
\usepackage[letterpaper,margin=0.75in]{geometry}
\usepackage[utf8]{inputenc}
\usepackage[T1]{fontenc}
\usepackage{hyperref}
%\usepackage[nodayofweek]{datetime}

%Hyperref format
\hypersetup{
	colorlinks=true,	% Hyperlinks are colored
	linkcolor=blue,		% Color of internal links (within document)
	citecolor=blue,	    % Color of internal links to the Reference page (within document)
	urlcolor=blue,		% Color of URLs (external) - default is magenta.
	pdftitle = {cover_letter},
	pdfsubject = {FCA},
	pdfauthor = {Gerardo A. Mazzei Capote},
	pdfkeywords = { },	
}
\usepackage{textcomp} % I think the bullet point shapes come from here
\usepackage[backend=bibtex, sorting=none,maxbibnames=99]{biblatex} %References are numbered per order of use in the text as opposed to alphabetically (default)
\addbibresource{BibTex/mypapers.bib}

\usepackage{tgpagella}% The headerrow command makes a table, this package
					  % Allows us to change the column seperation distance
\setlength{\tabcolsep}{0em} % Change the column seperation distance to 0em

\usepackage{enumitem} % Allows changing the spacing between items in a list
\setlist{nosep} % Change the spacing to zero

\usepackage{endnotes} % Enables the creation of end notes
\renewcommand{\notesname}{} % Remove the word "Notes" from the end notes
\let\enotesize\normalsize % Make endnotes the same size as document


% Create a length variable for the table in skills. It is the text width
% minus the indent
\newlength{\skillswidth}
\setlength{\skillswidth}{\textwidth}
\addtolength{\skillswidth}{-\parindent}

% indentsection style, used for sections that aren't already in lists
% that need indentation to the level of all text in the document
\newenvironment{indentsection}[1]%
{\begin{list}{}%
	{\setlength{\leftmargin}{#1}}%
	\item[]%
}
{\end{list}}

% opposite of above; bump a section back toward the left margin
\newenvironment{unindentsection}[1]%
{\begin{list}{}%
	{\setlength{\leftmargin}{-0.5#1}}%
	\item[]%
}
{\end{list}}

% Create a headerrow command for the header of each skills section
\newcommand{\headerrow}[3]
{\vspace{0.4em}
\noindent
% To get the three items spaced left, center, right, I had to use this funny
% \extracolsep stuff. Got it working though.
\begin{tabular*}{\textwidth}{l @{\extracolsep{\fill}} cr}
	\textbf{#1} & % Title/Postion
	#2 &		  % Company Name
	\textbf{#3}\\ % Employment dates
\end{tabular*}}

%\usepackage{bibentry}
%\nobibliography*

% and the actual content starts here
\begin{document}


\begin{center}
	{\LARGE \textbf{Gerardo Andrés Mazzei Capote}}

	45 N. Randall Ave, Apt. 109\ \ \textbullet
	\ \ Madison, WI- 53715, U.S.A.
	\\
	(608) 622-4643 \ \textbullet
	\ \ mazzeicapote@wisc.edu\\
	\href{https://www.linkedin.com/in/gerardo-mazzei-capote}{linkedin.com/in/gerardo-mazzei-capote}
	
\end{center}
\vspace{-0.5em}
\hrule
\vspace{0.4em}
\vspace{-1em}

\section*{COVER LETTER}
\vspace{0.4em}
\today
\\
To the talent acquisition team at Lilium:
\\
\\
\indent I am writing to apply for the position of Senior Material \& Processes Engineer - Thermoplastic Materials Engineer. Your vision to advance how the world moves through sustainable, air vehicles is extremely exciting and I would love to be a part of your talented team of engineers!

I am a passionate and curious mechanical engineer with 5 years of experience in polymer processing and polymer-based Additive manufacturing techniques. While I lack formal experience in aerospace, my graduate studies have made me well-versed in polymer science, failure prediction, and composite theory -- all attributes that align well with the requirements of the position. During the past five years, I have worked at the Polymer Engineering Center from the University of Wisconsin - Madison as a Research Assistant. My dissertation project involved developing strategies that allow engineers to safely assess the structural integrity of parts fabricated through additive manufacturing, a feat that required juggling diverse areas of knowledge, such as mechanics of materials, statistics, data analytics, machine learning, and polymer processing techniques. I additionally coordinated collaborative efforts that lead to peer reviewed publications and technical presentations. These included joint efforts between my University and international institutions, such as the Technical University of Munich, ENISE, RWTH Aachen, and companies like BMW, 3M, and Fused Form Corp. I also served as lab manager, where my responsibilities involved writing Standard Operating Procedures, and ensuring safety standards in the Polymer Engineering Center were up to the University's guidelines. Finally, my experience as a teaching assistant has made me a versatile communicator, having consistently received praise from my students for my ability to convey difficult concepts to the audience, or my capability to promptly respond to class inquiries in written form outside of class hours. 	 

I am looking forward to hearing from you in the future, and would love to have the opportunity to discuss my qualifications in more detail over an interview at your earliest convenience! Thank you for your time and consideration. 
Sincerely: 
\\
\begin{center}
Gerardo A. Mazzei Capote
\end{center}



\end{document}
