\documentclass[12pt,letterpaper]{article}
\usepackage[letterpaper,margin=0.75in]{geometry}
\usepackage[utf8]{inputenc}
\usepackage[T1]{fontenc}
\usepackage{hyperref}

%Hyperref format
\hypersetup{
	colorlinks=true,	% Hyperlinks are colored
	linkcolor=blue,		% Color of internal links (within document)
	citecolor=blue,	    % Color of internal links to the Reference page (within document)
	urlcolor=blue,		% Color of URLs (external) - default is magenta.
	pdftitle = {cover_letter},
	pdfsubject = {FCA},
	pdfauthor = {Gerardo A. Mazzei Capote},
	pdfkeywords = { },	
}
\usepackage{textcomp} % I think the bullet point shapes come from here
\usepackage[backend=bibtex, sorting=none,maxbibnames=99]{biblatex} %References are numbered per order of use in the text as opposed to alphabetically (default)
\addbibresource{BibTex/mypapers.bib}

\usepackage{tgpagella}% The headerrow command makes a table, this package
					  % Allows us to change the column seperation distance
\setlength{\tabcolsep}{0em} % Change the column seperation distance to 0em

\usepackage{enumitem} % Allows changing the spacing between items in a list
\setlist{nosep} % Change the spacing to zero

\usepackage{endnotes} % Enables the creation of end notes
\renewcommand{\notesname}{} % Remove the word "Notes" from the end notes
\let\enotesize\normalsize % Make endnotes the same size as document


% Create a length variable for the table in skills. It is the text width
% minus the indent
\newlength{\skillswidth}
\setlength{\skillswidth}{\textwidth}
\addtolength{\skillswidth}{-\parindent}

% indentsection style, used for sections that aren't already in lists
% that need indentation to the level of all text in the document
\newenvironment{indentsection}[1]%
{\begin{list}{}%
	{\setlength{\leftmargin}{#1}}%
	\item[]%
}
{\end{list}}

% opposite of above; bump a section back toward the left margin
\newenvironment{unindentsection}[1]%
{\begin{list}{}%
	{\setlength{\leftmargin}{-0.5#1}}%
	\item[]%
}
{\end{list}}

% Create a headerrow command for the header of each skills section
\newcommand{\headerrow}[3]
{\vspace{0.4em}
\noindent
% To get the three items spaced left, center, right, I had to use this funny
% \extracolsep stuff. Got it working though.
\begin{tabular*}{\textwidth}{l @{\extracolsep{\fill}} cr}
	\textbf{#1} & % Title/Postion
	#2 &		  % Company Name
	\textbf{#3}\\ % Employment dates
\end{tabular*}}

%\usepackage{bibentry}
%\nobibliography*

% and the actual content starts here
\begin{document}


\begin{center}
	{\LARGE \textbf{Gerardo Andrés Mazzei Capote}}

	45 N. Randall Ave, Apt. 109\ \ \textbullet
	\ \ Madison, WI- 53715, U.S.A.
	\\
	(608) 622-4643 \ \textbullet
	\ \ mazzeicapote@wisc.edu\\
	\href{https://www.linkedin.com/in/gerardo-mazzei-capote}{linkedin.com/in/gerardo-mazzei-capote}
	
\end{center}
\vspace{-0.5em}
\hrule
\vspace{0.4em}
\vspace{-1em}

\section*{COVER LETTER}
\vspace{0.4em}
September 23rd, 2020
\\
To the Hiring Manager of the R\&D Mechanical/Product Design team at P\&G:
\\
\\
The Oral B brand, as most P\&G brands, has an elevated reputation among consumers, and is considered the bastion of quality among dental health professionals. This degree of commitment and attention to detail aligns very well with my values and professional strengths. My name is Gerardo, and I am a soon to be graduated Mechanical Engineering PhD student with a lot of experience in design for Additive Manufacturing and polymer-processing technologies.

I am known between my peers as a highly communicative person who works great in a team. During my previous position as teaching and research assistant at the Polymer Engineering Center, my organizational and communicational skills, as well as my eagerness to learn quickly proved to be valuable assets to Prof. Tim Osswald, co-director of the institute. My efforts directly resulted in high ratings from students in multiple classes taught; technical presentations at world-renowned conferences of the likes of SFF, AMUG, and RAPID+TCT; and the generation of active funds for the institution through participation in multi-departmental grants and research ventures. I have also successfully managed undergraduate research assistants, and coordinated cooperative projects between universities, industry, and institutes, such as collaborations with BMW, Netzsch, the Technical University of Munich, and the University of Lyon - ENISE. Finally, my design expertise has resulted in the development of 3D printed heat exchangers, and a proof of concept COVID19 mask ---a project funded by the Wisconsin Alumni Research Foundation.

I truly believe that my experience with Additive Manufacturing technologies and knowledge of polymer processing techniques can aid in the creation and prototyping of new products for Oral B, and I would love to have the opportunity to form a part of your team of world-renowned professionals. If my profile interests you, it would be a privilege to discuss details of career opportunities whenever it is convenient for you. Thank you for your time, and I look forward to hearing from you.
\\
\\
Sincerely,
\\
\\
\begin{center}
Gerardo A. Mazzei Capote
\end{center}



\end{document}
