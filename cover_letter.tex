\documentclass[12pt,letterpaper]{article}
\usepackage[letterpaper,margin=0.75in]{geometry}
\usepackage[utf8]{inputenc}
\usepackage[T1]{fontenc}
\usepackage{hyperref}

%Hyperref format
\hypersetup{
	colorlinks=true,	% Hyperlinks are colored
	linkcolor=blue,		% Color of internal links (within document)
	citecolor=blue,	    % Color of internal links to the Reference page (within document)
	urlcolor=blue,		% Color of URLs (external) - default is magenta.
	pdftitle = {cover_letter},
	pdfsubject = {FCA},
	pdfauthor = {Gerardo A. Mazzei Capote},
	pdfkeywords = { },	
}
\usepackage{textcomp} % I think the bullet point shapes come from here
\usepackage[backend=bibtex, sorting=none,maxbibnames=99]{biblatex} %References are numbered per order of use in the text as opposed to alphabetically (default)
\addbibresource{BibTex/mypapers.bib}

\usepackage{tgpagella}% The headerrow command makes a table, this package
					  % Allows us to change the column seperation distance
\setlength{\tabcolsep}{0em} % Change the column seperation distance to 0em

\usepackage{enumitem} % Allows changing the spacing between items in a list
\setlist{nosep} % Change the spacing to zero

\usepackage{endnotes} % Enables the creation of end notes
\renewcommand{\notesname}{} % Remove the word "Notes" from the end notes
\let\enotesize\normalsize % Make endnotes the same size as document


% Create a length variable for the table in skills. It is the text width
% minus the indent
\newlength{\skillswidth}
\setlength{\skillswidth}{\textwidth}
\addtolength{\skillswidth}{-\parindent}

% indentsection style, used for sections that aren't already in lists
% that need indentation to the level of all text in the document
\newenvironment{indentsection}[1]%
{\begin{list}{}%
	{\setlength{\leftmargin}{#1}}%
	\item[]%
}
{\end{list}}

% opposite of above; bump a section back toward the left margin
\newenvironment{unindentsection}[1]%
{\begin{list}{}%
	{\setlength{\leftmargin}{-0.5#1}}%
	\item[]%
}
{\end{list}}

% Create a headerrow command for the header of each skills section
\newcommand{\headerrow}[3]
{\vspace{0.4em}
\noindent
% To get the three items spaced left, center, right, I had to use this funny
% \extracolsep stuff. Got it working though.
\begin{tabular*}{\textwidth}{l @{\extracolsep{\fill}} cr}
	\textbf{#1} & % Title/Postion
	#2 &		  % Company Name
	\textbf{#3}\\ % Employment dates
\end{tabular*}}

%\usepackage{bibentry}
%\nobibliography*

% and the actual content starts here
\begin{document}


\begin{center}
	{\LARGE \textbf{Gerardo Andrés Mazzei Capote}}

	45 N. Randall Ave, Apt. 109\ \ \textbullet
	\ \ Madison, WI- 53715, U.S.A.
	\\
	(608) 622-4643 \ \textbullet
	\ \ mazzeicapote@wisc.edu\\
	\href{https://www.linkedin.com/in/gerardo-mazzei-capote}{linkedin.com/in/gerardo-mazzei-capote}
	
\end{center}
\vspace{-0.5em}
\hrule
\vspace{0.4em}
\vspace{-1em}

\section*{COVER LETTER}
\vspace{0.4em}
September 28th, 2020
\\
To the Hiring Manager at 3D Systems:
\\
\\
3D systems' longevity and trajectory in the Additive Manufacturing field, as well as its drive to innovate and constantly evolve resonate with my personal values and pique my interest in your company, its vision, and mission. My name is Gerardo, and I am an eager engineer aiming to successfully apply to the position of Applied Research Engineer within your institution.

I am known between my peers as a highly versatile and independent research scientist who works great in a team. During my graduate studies, I had to tackle a variety of challenges related to fracture mechanics, data analysis, coding, polymer processing, and toolpathing. In my previous position as teaching and research assistant at the Polymer Engineering Center, my organizational and communicational skills, as well as my eagerness to learn quickly proved to be valuable assets to Prof. Tim Osswald, co-director of the institute. My efforts directly resulted in high ratings from students in multiple classes taught; technical presentations at world-renowned conferences of the likes of SFF, AMUG, and RAPID+TCT; and the generation of active funds for the institution through participation in multi-departmental grants, research ventures, and proposal submissions. I have also successfully managed undergraduate research assistants, and coordinated cooperative projects between universities, industry, and institutes, such as collaborations with BMW, Netzsch, the Technical University of Munich, and the University of Lyon - ENISE. Finally, my design expertise has resulted in the development of 3D printed heat exchangers, and a proof of concept COVID19 mask ---a project funded by the Wisconsin Alumni Research Foundation. Finally, I performed the function of lab manager of my research group, where I was in charge of ensuring adherence to safety guidelines, coordinating machine maintenance, and writing SOPs.

I truly believe that my expertise of Additive Manufacturing technologies and polymer processing techniques can aid the talented team at 3D Systems in creating physical products at a digital pace, and I would love to be given the opportunity to contribute to your enterprise. If my profile interests you, it would be a privilege to discuss details of career opportunities whenever it is convenient for you. Thank you for your time, and I look forward to hearing from you.
\\
\\
Sincerely,
\\
\\
\begin{center}
Gerardo A. Mazzei Capote
\end{center}



\end{document}
