\documentclass[12pt,letterpaper]{article}
\usepackage[letterpaper,margin=0.75in]{geometry}
\usepackage[utf8]{inputenc}
\usepackage[T1]{fontenc}
\usepackage{hyperref}
%\usepackage[nodayofweek]{datetime}

%Hyperref format
\hypersetup{
	colorlinks=true,	% Hyperlinks are colored
	linkcolor=blue,		% Color of internal links (within document)
	citecolor=blue,	    % Color of internal links to the Reference page (within document)
	urlcolor=blue,		% Color of URLs (external) - default is magenta.
	pdftitle = {cover_letter},
	pdfsubject = {FCA},
	pdfauthor = {Gerardo A. Mazzei Capote},
	pdfkeywords = { },	
}
\usepackage{textcomp} % I think the bullet point shapes come from here
\usepackage[backend=bibtex, sorting=none,maxbibnames=99]{biblatex} %References are numbered per order of use in the text as opposed to alphabetically (default)
\addbibresource{BibTex/mypapers.bib}

\usepackage{tgpagella}% The headerrow command makes a table, this package
					  % Allows us to change the column seperation distance
\setlength{\tabcolsep}{0em} % Change the column seperation distance to 0em

\usepackage{enumitem} % Allows changing the spacing between items in a list
\setlist{nosep} % Change the spacing to zero

\usepackage{endnotes} % Enables the creation of end notes
\renewcommand{\notesname}{} % Remove the word "Notes" from the end notes
\let\enotesize\normalsize % Make endnotes the same size as document


% Create a length variable for the table in skills. It is the text width
% minus the indent
\newlength{\skillswidth}
\setlength{\skillswidth}{\textwidth}
\addtolength{\skillswidth}{-\parindent}

% indentsection style, used for sections that aren't already in lists
% that need indentation to the level of all text in the document
\newenvironment{indentsection}[1]%
{\begin{list}{}%
	{\setlength{\leftmargin}{#1}}%
	\item[]%
}
{\end{list}}

% opposite of above; bump a section back toward the left margin
\newenvironment{unindentsection}[1]%
{\begin{list}{}%
	{\setlength{\leftmargin}{-0.5#1}}%
	\item[]%
}
{\end{list}}

% Create a headerrow command for the header of each skills section
\newcommand{\headerrow}[3]
{\vspace{0.4em}
\noindent
% To get the three items spaced left, center, right, I had to use this funny
% \extracolsep stuff. Got it working though.
\begin{tabular*}{\textwidth}{l @{\extracolsep{\fill}} cr}
	\textbf{#1} & % Title/Postion
	#2 &		  % Company Name
	\textbf{#3}\\ % Employment dates
\end{tabular*}}

%\usepackage{bibentry}
%\nobibliography*

% and the actual content starts here
\begin{document}


\begin{center}
	{\LARGE \textbf{Gerardo Andrés Mazzei Capote}}

	45 N. Randall Ave, Apt. 109\ \ \textbullet
	\ \ Madison, WI- 53715, U.S.A.
	\\
	(608) 622-4643 \ \textbullet
	\ \ mazzeicapote@wisc.edu\\
	\href{https://www.linkedin.com/in/gerardo-mazzei-capote}{linkedin.com/in/gerardo-mazzei-capote}
	
\end{center}
\vspace{-0.5em}
\hrule
\vspace{0.4em}
\vspace{-1em}

\section*{COVER LETTER}
\vspace{0.4em}
\today
\\
To the Hiring Managers at Virgin Hyperloop:
\\
\\
Evolving transportation to meet the demands of an ever more connected world is no small feat \textemdash particularly from an engineering standpoint. This technological challenge results extremely attractive to me, and I believe that my profile has the potential to greatly benefit the endeavors of Virgin Hyperloop. 

My undergraduate degree in Materials Engineering is supplemented by post-graduate studies in Mechanical Engineering, with a focus on polymer processing and Additive Manufacturing technologies. During my career, I have become, among other things, well-versed in mechanical and thermal evaluation techniques that allow me to produce informed decisions when it comes to material selection. I have leveraged this knowledge to improve the thermal performance of an Additive Manufactured metal/polymer Heat Exchanger, and increasing the dielectric constant of a thermoplastic material destined to produce topological crystal insulators.

My experience as a teaching assistant has greatly improved my skills as a communicator, consistently receiving high ratings from students. Additionally, I have been involved in a variety of collaborative projects where I have been tasked with supervising undergraduate students, and coordinating cooperative efforts with specialists from areas of expertise beyond my own, such as chemists and electrical engineers.

I truly believe that my profile would aid Virgin Hyperloop's mission of connecting the world, and would be more than happy to offer more detail pertaining to my qualifications, interests, and aptitudes in the form of an interview. Thank you for your time and consideration. Looking forward to hearing back from you:  
\\
\begin{center}
Gerardo A. Mazzei Capote
\end{center}



\end{document}
