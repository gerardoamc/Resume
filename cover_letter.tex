\documentclass[12pt,letterpaper]{article}
\usepackage[letterpaper,margin=0.75in]{geometry}
\usepackage[utf8]{inputenc}
\usepackage[T1]{fontenc}
\usepackage{hyperref}
%\usepackage[nodayofweek]{datetime}

%Hyperref format
\hypersetup{
	colorlinks=true,	% Hyperlinks are colored
	linkcolor=blue,		% Color of internal links (within document)
	citecolor=blue,	    % Color of internal links to the Reference page (within document)
	urlcolor=blue,		% Color of URLs (external) - default is magenta.
	pdftitle = {cover_letter},
	pdfsubject = {FCA},
	pdfauthor = {Gerardo A. Mazzei Capote},
	pdfkeywords = { },	
}
\usepackage{textcomp} % I think the bullet point shapes come from here
\usepackage[backend=bibtex, sorting=none,maxbibnames=99]{biblatex} %References are numbered per order of use in the text as opposed to alphabetically (default)
\addbibresource{BibTex/mypapers.bib}

\usepackage{tgpagella}% The headerrow command makes a table, this package
					  % Allows us to change the column seperation distance
\setlength{\tabcolsep}{0em} % Change the column seperation distance to 0em

\usepackage{enumitem} % Allows changing the spacing between items in a list
\setlist{nosep} % Change the spacing to zero

\usepackage{endnotes} % Enables the creation of end notes
\renewcommand{\notesname}{} % Remove the word "Notes" from the end notes
\let\enotesize\normalsize % Make endnotes the same size as document


% Create a length variable for the table in skills. It is the text width
% minus the indent
\newlength{\skillswidth}
\setlength{\skillswidth}{\textwidth}
\addtolength{\skillswidth}{-\parindent}

% indentsection style, used for sections that aren't already in lists
% that need indentation to the level of all text in the document
\newenvironment{indentsection}[1]%
{\begin{list}{}%
	{\setlength{\leftmargin}{#1}}%
	\item[]%
}
{\end{list}}

% opposite of above; bump a section back toward the left margin
\newenvironment{unindentsection}[1]%
{\begin{list}{}%
	{\setlength{\leftmargin}{-0.5#1}}%
	\item[]%
}
{\end{list}}

% Create a headerrow command for the header of each skills section
\newcommand{\headerrow}[3]
{\vspace{0.4em}
\noindent
% To get the three items spaced left, center, right, I had to use this funny
% \extracolsep stuff. Got it working though.
\begin{tabular*}{\textwidth}{l @{\extracolsep{\fill}} cr}
	\textbf{#1} & % Title/Postion
	#2 &		  % Company Name
	\textbf{#3}\\ % Employment dates
\end{tabular*}}

%\usepackage{bibentry}
%\nobibliography*

% and the actual content starts here
\begin{document}


\begin{center}
	{\LARGE \textbf{Gerardo Andrés Mazzei Capote}}

	45 N. Randall Ave, Apt. 109\ \ \textbullet
	\ \ Madison, WI- 53715, U.S.A.
	\\
	(608) 622-4643 \ \textbullet
	\ \ mazzeicapote@wisc.edu\\
	\href{https://www.linkedin.com/in/gerardo-mazzei-capote}{linkedin.com/in/gerardo-mazzei-capote}
	
\end{center}
\vspace{-0.5em}
\hrule
\vspace{0.4em}
\vspace{-1em}

\section*{COVER LETTER}
\vspace{0.4em}
\today
\\
To whom it may concern:
\\
\\
I am known between my peers as a highly versatile, proactive, and responsible research scientist who can work great in a team, and I have the firm belief that my background using an ABB 6 axis robot, and my skills in experimental design, CAD, and statistical analysis would benefit greatly the R\&D Mechanical Engineering position offered by your company.

My excellent written and oral communication skills have resulted directly in high ratings from students in classes taught while working as a Teaching Assistant; technical presentations at world-renowned conferences of the likes of SFF, AMUG, and RAPID+TCT; peer-reviewed publications in journals like the Additive Manufacturing journal; and the generation of active funds for the institution through preparation of multi-departmental grants, research ventures, and NSF proposal submissions. I have also successfully managed undergraduate research assistants, and coordinated cooperative projects between universities, industry, and institutes, including the likes of BMW, Netzsch, the Technical University of Munich, and the University of Lyon - ENISE.

I believe that my aptitudes can prove beneficial for this position, and I am very eager to demonstrate why over an interview. If my profile interests you, it would be a privilege to discuss details of career opportunities. Thank you for your time, and I look forward to hearing from you. Have a great rest of your day,
\\
\\
\begin{center}
Gerardo A. Mazzei Capote
\end{center}



\end{document}
