\documentclass[12pt,letterpaper]{article}
\usepackage[letterpaper,margin=0.75in]{geometry}
\usepackage[utf8]{inputenc}
\usepackage[T1]{fontenc}
\usepackage{hyperref}
%\usepackage[nodayofweek]{datetime}

%Hyperref format
\hypersetup{
	colorlinks=true,	% Hyperlinks are colored
	linkcolor=blue,		% Color of internal links (within document)
	citecolor=blue,	    % Color of internal links to the Reference page (within document)
	urlcolor=blue,		% Color of URLs (external) - default is magenta.
	pdftitle = {cover_letter},
	pdfsubject = {FCA},
	pdfauthor = {Gerardo A. Mazzei Capote},
	pdfkeywords = { },	
}
\usepackage{textcomp} % I think the bullet point shapes come from here
\usepackage[backend=bibtex, sorting=none,maxbibnames=99]{biblatex} %References are numbered per order of use in the text as opposed to alphabetically (default)
\addbibresource{BibTex/mypapers.bib}

\usepackage{tgpagella}% The headerrow command makes a table, this package
					  % Allows us to change the column seperation distance
\setlength{\tabcolsep}{0em} % Change the column seperation distance to 0em

\usepackage{enumitem} % Allows changing the spacing between items in a list
\setlist{nosep} % Change the spacing to zero

\usepackage{endnotes} % Enables the creation of end notes
\renewcommand{\notesname}{} % Remove the word "Notes" from the end notes
\let\enotesize\normalsize % Make endnotes the same size as document


% Create a length variable for the table in skills. It is the text width
% minus the indent
\newlength{\skillswidth}
\setlength{\skillswidth}{\textwidth}
\addtolength{\skillswidth}{-\parindent}

% indentsection style, used for sections that aren't already in lists
% that need indentation to the level of all text in the document
\newenvironment{indentsection}[1]%
{\begin{list}{}%
	{\setlength{\leftmargin}{#1}}%
	\item[]%
}
{\end{list}}

% opposite of above; bump a section back toward the left margin
\newenvironment{unindentsection}[1]%
{\begin{list}{}%
	{\setlength{\leftmargin}{-0.5#1}}%
	\item[]%
}
{\end{list}}

% Create a headerrow command for the header of each skills section
\newcommand{\headerrow}[3]
{\vspace{0.4em}
\noindent
% To get the three items spaced left, center, right, I had to use this funny
% \extracolsep stuff. Got it working though.
\begin{tabular*}{\textwidth}{l @{\extracolsep{\fill}} cr}
	\textbf{#1} & % Title/Postion
	#2 &		  % Company Name
	\textbf{#3}\\ % Employment dates
\end{tabular*}}

%\usepackage{bibentry}
%\nobibliography*

% and the actual content starts here
\begin{document}


\begin{center}
	{\LARGE \textbf{Gerardo Andrés Mazzei Capote}}

	45 N. Randall Ave, Apt. 109\ \ \textbullet
	\ \ Madison, WI- 53715, U.S.A.
	\\
	(608) 622-4643 \ \textbullet
	\ \ mazzeicapote@wisc.edu\\
	\href{https://www.linkedin.com/in/gerardo-mazzei-capote}{linkedin.com/in/gerardo-mazzei-capote}
	
\end{center}
\vspace{-0.5em}
\hrule
\vspace{0.4em}
\vspace{-1em}

\section*{COVER LETTER}
\vspace{0.4em}
\today
\\
To the talent acquisition team at Boston Consulting Group:
\\
\\
\indent I am writing to apply for the position of Full-time Consultant. I am a versatile and curious mechanical engineer with proven capacity to do research, and communicate complex ideas effectively -- whether in written or oral form. I was drawn to the Boston Consulting Group given its proven record of solving problems and offering insights to a wide range of industries. I love the prospect of potentially learning something new with every single client, and leveraging my skill set to effectively solve a problem. I also liked the excitement all of your colleagues displayed at all the ADC virtual events I have attended in the recent past. It really makes BCG feel like an amazing place to work, with lots of room for personal and professional growth!

My academic and professional experiences have given me the interpersonal and analytical skills to succeed at BCG. During the past five years, I have worked at the Polymer Engineering Center from the University of Wisconsin - Madison as a Research Assistant. My dissertation project involved developing strategies that allow engineers to safely assess the structural integrity of parts fabricated through additive manufacturing, a feat that required juggling diverse areas of knowledge, such as mechanics of materials, statistics, data analytics, machine learning, and polymer processing techniques. I additionally coordinated collaborative efforts that lead to peer reviewed publications and technical presentations. These included joint efforts between my University and international institutions, such as the Technical University of Munich, ENISE, RWTH Aachen, and foreign companies like BMW and Fused Form Corp.

I also served as Vice-president for the Madison chapter of the Society of Plastic Engineers, from August 2018 to May 2020. During this time, my teammates and I arranged industry visits to the facilities of Midwest Prototyping in Blue Mounds and 3M in Maplewood, as well as a variety of social events aimed at increasing the interest of undergraduate students in the field of polymer processing. Finally, my experience as a Teaching Assistant has made me an effective communicator, having consistently received high ratings from my students at the end of the semesters where I taught a class.	 

I am looking forward to hearing from you in the future, and would love to have the opportunity to discuss my qualifications in more detail over an interview at your earliest convenience! Thank you for your time and consideration. 
Sincerely: 
\\
\begin{center}
Gerardo A. Mazzei Capote
\end{center}



\end{document}
