\documentclass[11pt,letterpaper]{article}
\usepackage[letterpaper,margin=0.75in]{geometry}
\usepackage[utf8]{inputenc}
\usepackage[T1]{fontenc}
\usepackage{hyperref}
%Hyperref format
\hypersetup{
	colorlinks=true,	% Hyperlinks are colored
	linkcolor=blue,		% Color of internal links (within document)
	citecolor=blue,	    % Color of internal links to the Reference page (within document)
	urlcolor=blue,		% Color of URLs (external) - default is magenta.
	pdftitle = {Predicting mechanical properties of FFF parts1},
	pdfsubject = {Prelim 2020},
	pdfauthor = {Gerardo A. Mazzei Capote},
	pdfkeywords = { },	
}
\usepackage{textcomp} % I think the bullet point shapes come from here
\usepackage[backend=bibtex, sorting=none,maxbibnames=99]{biblatex} %References are numbered per order of use in the text as opposed to alphabetically (default)
\addbibresource{BibTex/mypapers.bib}

\usepackage{tgpagella}% The headerrow command makes a table, this package
					  % Allows us to change the column seperation distance
\setlength{\tabcolsep}{0em} % Change the column seperation distance to 0em

\usepackage{enumitem} % Allows changing the spacing between items in a list
\setlist{nosep} % Change the spacing to zero

\usepackage{endnotes} % Enables the creation of end notes
\renewcommand{\notesname}{} % Remove the word "Notes" from the end notes
\let\enotesize\normalsize % Make endnotes the same size as document

\usepackage{fancyhdr} % Used to creat headers and footers

% Create a page style for the first page
\fancypagestyle{firststyle}
{
	\fancyhf{} % clear the header settings
	\renewcommand{\headrulewidth}{0pt} % Remove the horizontal line
	\fancyfoot[C]{\normalsize continued} % Add the word "continued" at normal size to the footer
}

% Create page style for the header of the second page, not currently used
\fancypagestyle{secondstyle}{
	\fancyhf{}
	\renewcommand{\headrulewidth}{0pt} % remove the horizontal line in the header
	\fancyhead[C]{\LARGE \textbf{Gerardo Andrés Mazzei Capote} \\
				\normalsize Resume -- page 2}
}

% Create a length variable for the table in skills. It is the text width
% minus the indent
\newlength{\skillswidth}
\setlength{\skillswidth}{\textwidth}
\addtolength{\skillswidth}{-\parindent}

% indentsection style, used for sections that aren't already in lists
% that need indentation to the level of all text in the document
\newenvironment{indentsection}[1]%
{\begin{list}{}%
	{\setlength{\leftmargin}{#1}}%
	\item[]%
}
{\end{list}}

% opposite of above; bump a section back toward the left margin
\newenvironment{unindentsection}[1]%
{\begin{list}{}%
	{\setlength{\leftmargin}{-0.5#1}}%
	\item[]%
}
{\end{list}}

% Create a headerrow command for the header of each skills section
\newcommand{\headerrow}[3]
{\vspace{0.4em}
\noindent
% To get the three items spaced left, center, right, I had to use this funny
% \extracolsep stuff. Got it working though.
\begin{tabular*}{\textwidth}{l @{\extracolsep{\fill}} cr}
	\textbf{#1} & % Title/Postion
	#2 &		  % Company Name
	\textbf{#3}\\ % Employment dates
\end{tabular*}}

%\usepackage{bibentry}
%\nobibliography*

% and the actual content starts here
\begin{document}
	% Include this so "continue" is printed in the first page footer
	\thispagestyle{firststyle}

% SECOND PAGE
	
\begin{center}
	{\LARGE \textbf{Gerardo Andrés Mazzei Capote}}

	45 N. Randall Ave, Apt. 109\ \ \textbullet
	\ \ Madison, WI- 53715
	\\
	(608) 622-4643 \ \textbullet
	\ \ mazzeicapote@wisc.edu\\
	\href{https://www.linkedin.com/in/gerardo-mazzei-capote}{linkedin.com/in/gerardo-mazzei-capote}
	
\end{center}

\vspace{-1em}

\subsection*{SUMMARY}
	\vspace{-0.5em}
	\hrule
	\vspace{0.4em}
	\begin{indentsection}{\parindent}
	Detail-oriented mechanical engineer with a minor in business. Experienced in polymer-based additive manufacturing technologies and polymer processing. Highly skilled in composite theory, failure criteria, thermal analysis, and mechanical testing of polymers. Highly skilled at communicating.
	\end{indentsection}


\subsection*{EDUCATION}
	\vspace{-0.5em}
	\hrule
	\vspace{0.4em}
	\begin{itemize}
	\item
	\textbf{University of Wisconsin-Madison} - \emph{Madison, WI, U.S.A.}
	\begin{itemize}	
	\item
	\textbf{PhD}: Mechanical Engineering --- \emph{2018 - present} (Expected May 2021) 
	\item
	\textbf{MSc}: Mechanical Engineering --- \emph{2016 - 2018}. 
	%\item
	%GPA: 3.813 / 4.
\end{itemize}

	\item
	\textbf{Universidad Simón Bolívar} - \emph{Caracas, Venezuela}
	\begin{itemize}	
		\item
		\textbf{BSc}: Materials Engineering --- \emph{2009 - 2016}. 
		%\item
		%GPA: 3.9375 / 5.
	\end{itemize}
	\end{itemize}

\subsection*{ENGINEERING EXPERIENCE}
	\vspace{-0.5em}
	\hrule

	\headerrow
		{PhD Candidate}
		{under Prof. Tim Osswald, UW-Madison}
		{August 2016 - Present}
		
	\begin{itemize}
		\item Extruded a customized ABS filament with tight dimensional tolerances to achieve high precision volumetric output during 3D printing.
		\item Predicted part failure of 3D printed parts using a failure criterion that includes stress interactions.
		\item Developed and produced 3D printed coupons with unusual bead orientations using a customized 6-axis robotic printer.
		\item Collaborated with a company to develop a 3D printer with in-line sensors that capture processing parameter data in real time.
		\item Supervised the extrusion of Polyethylene Terephthalate (PET) filament produced with discarded bottles as the parent material.
		\item Developed and manufactured a low-cost, reusable N9X concept mask during the COVID19 pandemic. 
	\end{itemize}

\subsection*{SKILLS}
\vspace{-0.5em}
\hrule
\vspace{0.4em}

\textbf{Polymer processing techniques}: Extrusion, Injection Molding, FDM/FFF, SLS, DLS\\
\textbf{Analysis techniques}: TGA, DSC, LFA, DMA, Destructive Mechanical Testing, $\mu$CT\\
\textbf{Programming languages}: MATLAB, Python, R, RAPID, G-code\\
\textbf{Engineering software}: Solidworks, EES, Origin, Jupyter Notebooks\\
\textbf{Other software}: Microsoft Office Suite, Adobe Animate\\
\textbf{Languages}: English, Spanish, Portuguese\\


\subsection*{OTHER EXPERIENCE}

\vspace{-0.5em}
\hrule
	
	\headerrow
		{Teaching Assistant}
		{University of Wisconsin-Madison} 
		{August 2017 - Present}
		Responsible for the instruction and grading of `ME370 - Energy Systems Lab`, `ME514 - Additive Manufacturing`, and `ME418 - Engineering Design with Polymers`, all offered by the Mechanical Engineering Department. 
	
	\headerrow
		{Vice President}
		{Society of Plastic Engineers - Madison Chapter} 
		{August 2018- May 2020}
		Coordinated industry visits and outreach activities aimed at increasing the interest of engineering students in the field of polymer processing.
	
\pagebreak % Make sure the first page ends here
\thispagestyle{empty} % clear the style so "continue" isn't printed

% Header for the second page. This could be in the header but
% I didn't think that looked good and didn't need the space.
\begin{center}
	\LARGE \textbf{Gerardo Andrés Mazzei Capote} \\
	\normalsize Resume -- page 2
\end{center}

\subsection*{TECHNICAL PRESENTATIONS}

	\vspace{-0.5em}
	\hrule
	\headerrow
		{AMUG}
		{Chicago, IL}
		{2019}
	\begin{itemize}
		\item \emph{A Novel Failure Criterion Applied for Fused Filament Fabrication Parts}.
	\end{itemize}

	\headerrow
		{RAPID}{Fort Worth, TX}{2018}
	\begin{itemize}
		\item \emph{A Tensor Based Failure Criterion for FFF Manufactured Parts}.
	\end{itemize}

	\headerrow
		{SFF}
		{Austin, TX}
		{2017}
	\begin{itemize}
		\item \emph{Towards a Robust Production of FFF End-User Parts with Improved Tensile Properties}. 
	\end{itemize}

\subsection*{PUBLICATIONS}
	\vspace{-0.5em}
	\hrule
	\vspace{0.6em}

\begin{enumerate}
	\item \fullcite{Pfeifer2016}
	\item \fullcite{MazzeiCapote2017}
	\item \fullcite{MazzeiCapote2019}
	\item \fullcite{Colon2019}
	\item \fullcite{MazzeiJCompSci}
\end{enumerate}
	
\subsection*{ADDITIONAL INFORMATION}
	\vspace{-0.5em}
	\hrule
	\vspace{0.8em}
	\begin{itemize}
		\item Exchange student through the Rotary Youth Exchange Program (August 2008 to June 2009).
		\item Venezuelan and Italian citizenship.
		\item Holder of Brazilian permanent resident visa.
	\end{itemize}
	
	
\end{document}
