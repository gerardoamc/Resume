\documentclass[11pt,letterpaper]{article}
\usepackage[letterpaper,margin=0.75in]{geometry}
\usepackage[utf8]{inputenc}
\usepackage[T1]{fontenc}
\usepackage{hyperref}
%Hyperref format
\hypersetup{
	colorlinks=true,	% Hyperlinks are colored
	linkcolor=blue,		% Color of internal links (within document)
	citecolor=blue,	    % Color of internal links to the Reference page (within document)
	urlcolor=blue,		% Color of URLs (external) - default is magenta.
	pdftitle = {Resume},
	pdfsubject = {CV Gerardo Mazzei},
	pdfauthor = {Gerardo A. Mazzei Capote},
	pdfkeywords = { },	
}
\usepackage{textcomp} % I think the bullet point shapes come from here
\usepackage[backend=bibtex, sorting=none,maxbibnames=99]{biblatex} %References are numbered per order of use in the text as opposed to alphabetically (default)
\addbibresource{BibTex/mypapers.bib}

\usepackage{tgpagella}% The headerrow command makes a table, this package
					  % Allows us to change the column seperation distance
\setlength{\tabcolsep}{0em} % Change the column seperation distance to 0em

\usepackage{enumitem} % Allows changing the spacing between items in a list
\setlist{nosep} % Change the spacing to zero

\usepackage{endnotes} % Enables the creation of end notes
\renewcommand{\notesname}{} % Remove the word "Notes" from the end notes
\let\enotesize\normalsize % Make endnotes the same size as document

\usepackage{fancyhdr} % Used to creat headers and footers

% Create a page style for the first page
\fancypagestyle{firststyle}
{
	\fancyhf{} % clear the header settings
	\renewcommand{\headrulewidth}{0pt} % Remove the horizontal line
	\fancyfoot[C]{\normalsize continued} % Add the word "continued" at normal size to the footer
}

% Create page style for the header of the second page, not currently used
\fancypagestyle{secondstyle}{
	\fancyhf{}
	\renewcommand{\headrulewidth}{0pt} % remove the horizontal line in the header
	\fancyhead[C]{\LARGE \textbf{Gerardo Andrés Mazzei Capote} \\
				\normalsize Resume -- page 2}
}

% Create a length variable for the table in skills. It is the text width
% minus the indent
\newlength{\skillswidth}
\setlength{\skillswidth}{\textwidth}
\addtolength{\skillswidth}{-\parindent}

% indentsection style, used for sections that aren't already in lists
% that need indentation to the level of all text in the document
\newenvironment{indentsection}[1]%
{\begin{list}{}%
	{\setlength{\leftmargin}{#1}}%
	\item[]%
}
{\end{list}}

% opposite of above; bump a section back toward the left margin
\newenvironment{unindentsection}[1]%
{\begin{list}{}%
	{\setlength{\leftmargin}{-0.5#1}}%
	\item[]%
}
{\end{list}}

% Create a headerrow command for the header of each skills section
\newcommand{\headerrow}[3]
{\vspace{0.4em}
\noindent
% To get the three items spaced left, center, right, I had to use this funny
% \extracolsep stuff. Got it working though.
\begin{tabular*}{\textwidth}{l @{\extracolsep{\fill}} cr}
	\textbf{#1} & % Title/Postion
	#2 &		  % Company Name
	\textbf{#3}\\ % Employment dates
\end{tabular*}}

%\usepackage{bibentry}
%\nobibliography*

% and the actual content starts here
\begin{document}
	% Include this so "continue" is printed in the first page footer
	\thispagestyle{firststyle}

% SECOND PAGE
	
\begin{center}
	{\LARGE \textbf{Gerardo Andrés Mazzei Capote}}

	45 N. Randall Ave, Apt. 109\ \ \textbullet
	\ \ Madison, WI- 53715
	\\
	(608) 622-4643 \ \textbullet
	\ \ mazzeicapote@wisc.edu\\
	\href{https://www.linkedin.com/in/gerardo-mazzei-capote}{linkedin.com/in/gerardo-mazzei-capote}\\
	\href{https://orcid.org/0000-0002-1951-6600}{ORCID: 0000-0002-1951-6600}
	
\end{center}

\vspace{-1em}

\subsection*{SUMMARY}
	\vspace{-0.5em}
	\hrule
	\vspace{0.4em}
	\begin{indentsection}{\parindent}
	Mechanical engineer with a minor in business. Specialized in polymer additive manufacturing technologies and polymer processing. Highly skilled in thermal analysis techniques, mechanical testing, composite theory, and failure criteria. Versatile communicator fluent in three languages.
	\end{indentsection}


\subsection*{EDUCATION}
	\vspace{-0.5em}
	\hrule
	\vspace{0.4em}
	\textbf{University of Wisconsin-Madison} - \emph{Madison, WI, U.S.A.}
	\begin{itemize}	
	\item
	\textbf{PhD}: Mechanical Engineering --- \emph{2018 - present} (Expected September 2021) 
	\item
	\textbf{MSc}: Mechanical Engineering --- \emph{2016 - 2018}. 
	%GPA: 3.813 / 4.
	\end{itemize}

	\noindent \textbf{Universidad Simón Bolívar} - \emph{Caracas, Venezuela}
	\begin{itemize}	
	\item
	\textbf{BSc}: Materials Engineering --- \emph{2009 - 2016}. 
	%GPA: 3.9375 / 5.
	\end{itemize}
	%\end{itemize}

\subsection*{PROFESSIONAL EXPERIENCE}
	\vspace{-0.5em}
	\hrule

	\headerrow
		{Research Assistant}
		{Polymer Engineering Center, UW-Madison}
		{August 2016 - Present}		

		\noindent -- PhD Thesis --- \emph{Predicting Mechanical Properties of Fused Filament Fabrication (FFF) Parts}
		\begin{itemize}
			\item Predicted structural integrity of FFF parts through the use of criteria.
			\item Deployed a Machine Learning system capable of predicting mechanical properties of FFF parts based on slicing parameters and in-line measurements of process indicators.
			\item Extruded ABS based thermoplastic filament within 2\% of target geometric values. 
		\end{itemize}
		
		\noindent -- MSc Thesis --- \emph{Defining a Failure Surface for Fused Filament Fabrication Parts Using a Novel Failure Criterion}
		\begin{itemize}
			\item Constructed a failure surface for Fused Filament Fabrication (FFF) parts, allowing part failure prediction if the mechanical requirements of the part are known.
			\item Designed custom test specimens based on mechanics of materials.
			\item Implemented toolpath solutions to produce test specimens with out of ordinary bead orientations using a 6-axis robotic printer.
		\end{itemize}
		
		\noindent -- Production of Topological Crystal Insulators
		
		\begin{itemize}
			\item Collaborated with chemists and electrical engineers to refine the design of topological crystal insulators to improve manufacturability.
			\item Compounded a high dielectric thermoplastic material using a twin screw extrusion system.
			\item Implemented toolpath solutions to produce crystal insulators using diverse Additive Manufacturing techniques, such as Fused Filament Fabrication and Digital Light Synthesis.  
		\end{itemize}
				
		\noindent -- Extrusion of recycled PET filament for Fused Filament Fabrication
		
		\begin{itemize}
			\item Extruded a thermoplastic filament using recycled PET flakes as the parent material. Geometric tolerances within 2\% of desired values.
		\end{itemize}

\subsection*{SKILLS}
\vspace{-0.5em}
\hrule
\vspace{0.4em}

\textbf{Manufacturing techniques}: Extrusion, Injection Molding, FDM/FFF, SLS, DLS\\
\textbf{Analysis techniques}: TGA, DSC, LFA, DMA, FTIR, Mechanical Testing, $\mu$CT, Metallography\\
\textbf{Programming languages}: MATLAB, Python, R, RAPID, G-code\\
\textbf{Engineering software}: Solidworks, EES, Origin, Moldflow, ANSYS\\
\textbf{Other software}: Microsoft Office Suite, Adobe Animate\\
\textbf{Languages}: English, Spanish, Portuguese\\

\pagebreak
\thispagestyle{empty} % clear the style so "continue" isn't printed

% Header for the second page. This could be in the header but
% I didn't think that looked good and didn't need the space.
\begin{center}
	\LARGE \textbf{Gerardo Andrés Mazzei Capote} \\
	\normalsize Resume -- page 2
\end{center}

\subsection*{OTHER PROFESSIONAL EXPERIENCE}

\vspace{-0.5em}
\hrule

\headerrow
{Lab Manager}
{Polymer Engineering Center, UW-Madison} 
{August 2017 - May 2019}
In charge of ordering lab supplies, ensuring proper adherence to safety guidelines dictated by the university, and coordinating installation and maintenance of machinery and lab equipment.

\headerrow
{Teaching Assistant}
{Mech. Eng. Department, UW-Madison} 
{August 2017 - December 2020}
Responsible for the instruction and grading of `ME370 - Energy Systems Lab`, `ME514 - Additive Manufacturing`, and `ME418 - Engineering Design with Polymers`, all offered by the Mechanical Engineering Department. Consistently received high ratings from students.

\headerrow
{Vice President}
{Society of Plastic Engineers - Madison Chapter} 
{August 2018 - May 2020}
Coordinated industry visits and outreach activities aimed at increasing the interest of engineering students in the field of polymer processing. Highlights include visits to the 3M campus in the Twin Cities - MN, and to the Trek facilities in Waterloo - WI. 

\subsection*{TECHNICAL PRESENTATIONS}
\vspace{-0.5em}
\hrule
\vspace{0.6em}

\begin{enumerate}
	\item \emph{A Novel Failure Criteria Applied for Fused Filament Fabrication Parts}. AMUG - Chicago, IL - 2019. 
	\item \emph{A Tensor Based Failure Criterion for FFF Manufactured Parts}.  RAPID - Fort Worth, TX - 2018.
	\item \emph{Towards a Robust Production of FFF End-User Parts with Improved Tensile Properties}. SFF - Austin, TX - 2017. 

\end{enumerate}


\subsection*{PUBLICATIONS}
	\vspace{-0.5em}
	\hrule
	\vspace{0.6em}

\begin{enumerate}
	\item \fullcite{Pfeifer2016}
	\item \fullcite{MazzeiCapote2017}
	\item \fullcite{MazzeiCapote2019}
	\item \fullcite{Colon2019}
	\item \fullcite{MazzeiJCompSci}
	\item \fullcite{Osswald2020a}
	\item \fullcite{ColonQuintana2021}
	\item \fullcite{MazzeiCapote2021}
\end{enumerate}
	
%\subsection*{ADDITIONAL INFORMATION}
	%\vspace{-0.5em}
	%\hrule
	%\vspace{0.8em}
	%\begin{itemize}
		%\item Venezuelan and Italian citizenship.
		%\item Holder of Brazilian permanent resident visa.
		%\item Exchange student through the Rotary Youth Exchange Program (August 2008 to June 2009).
	%\end{itemize}
	
	
\end{document}
